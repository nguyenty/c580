\documentclass{article}\usepackage[]{graphicx}\usepackage[]{color}
%% maxwidth is the original width if it is less than linewidth
%% otherwise use linewidth (to make sure the graphics do not exceed the margin)
\makeatletter
\def\maxwidth{ %
  \ifdim\Gin@nat@width>\linewidth
    \linewidth
  \else
    \Gin@nat@width
  \fi
}
\makeatother

\definecolor{fgcolor}{rgb}{0.345, 0.345, 0.345}
\newcommand{\hlnum}[1]{\textcolor[rgb]{0.686,0.059,0.569}{#1}}%
\newcommand{\hlstr}[1]{\textcolor[rgb]{0.192,0.494,0.8}{#1}}%
\newcommand{\hlcom}[1]{\textcolor[rgb]{0.678,0.584,0.686}{\textit{#1}}}%
\newcommand{\hlopt}[1]{\textcolor[rgb]{0,0,0}{#1}}%
\newcommand{\hlstd}[1]{\textcolor[rgb]{0.345,0.345,0.345}{#1}}%
\newcommand{\hlkwa}[1]{\textcolor[rgb]{0.161,0.373,0.58}{\textbf{#1}}}%
\newcommand{\hlkwb}[1]{\textcolor[rgb]{0.69,0.353,0.396}{#1}}%
\newcommand{\hlkwc}[1]{\textcolor[rgb]{0.333,0.667,0.333}{#1}}%
\newcommand{\hlkwd}[1]{\textcolor[rgb]{0.737,0.353,0.396}{\textbf{#1}}}%

\usepackage{framed}
\makeatletter
\newenvironment{kframe}{%
 \def\at@end@of@kframe{}%
 \ifinner\ifhmode%
  \def\at@end@of@kframe{\end{minipage}}%
  \begin{minipage}{\columnwidth}%
 \fi\fi%
 \def\FrameCommand##1{\hskip\@totalleftmargin \hskip-\fboxsep
 \colorbox{shadecolor}{##1}\hskip-\fboxsep
     % There is no \\@totalrightmargin, so:
     \hskip-\linewidth \hskip-\@totalleftmargin \hskip\columnwidth}%
 \MakeFramed {\advance\hsize-\width
   \@totalleftmargin\z@ \linewidth\hsize
   \@setminipage}}%
 {\par\unskip\endMakeFramed%
 \at@end@of@kframe}
\makeatother

\definecolor{shadecolor}{rgb}{.97, .97, .97}
\definecolor{messagecolor}{rgb}{0, 0, 0}
\definecolor{warningcolor}{rgb}{1, 0, 1}
\definecolor{errorcolor}{rgb}{1, 0, 0}
\newenvironment{knitrout}{}{} % an empty environment to be redefined in TeX

\usepackage{alltt}
\usepackage[margin=0.5in]{geometry}
\usepackage{boxedminipage}
\usepackage{amsfonts}
\usepackage{amsmath} 
\usepackage{amssymb}
\usepackage{graphicx}
\usepackage{amsthm}
\usepackage{t1enc}
\usepackage{subfig}
\usepackage{hyperref}
\IfFileExists{upquote.sty}{\usepackage{upquote}}{}
\begin{document}
\title{\color{red}HW3 - STAT 580 - Sp 2015}
\author{Yet T Nguyen}
\date{02/26/2015}
\maketitle

\section*{1.}
\begin{itemize}
\item [(a)] 
\begin{align*}
P(U \leq r(X)) & = \int_\mathbb{X} \left(\int_0^{r(x)} du\right) g(x) dx\\
               & = \int_\mathbb{X} r(x)g(x)dx\\
               & = \int_\mathbb{X}\frac{q(x)}{\alpha g(x)}g(x) dx \\
               & = \frac{1}{\alpha}\int_\mathbb{X}q(x)dx.
\end{align*}


\item [(b)] 
\begin{align*}
P(X\in A, U \leq r(X)) & = \int_{A} \left(\int_0^{r(x)} du\right) g(x) dx\\
               & = \int_{A} r(x)g(x)dx\\
               & = \int_{A}\frac{q(x)}{\alpha g(x)}g(x) dx \\
               & = \frac{1}{\alpha}\int_{A}q(x)dx.
\end{align*}

Therefore, 

\begin{align*}
P(Y\in  A) &= P(X\in A|U\leq r(X)) \\
           &= \frac{P(X\in A, U \leq r(X))}{P(U \leq r(X))}\\
           &=\frac{\frac{1}{\alpha}\int_{A}q(x)dx}{\frac{1}{\alpha}\int_\mathbb{X}q(x)dx}\\
           &=\frac{\frac{1}{\alpha}\int_{A}cq(x)dx}{\frac{1}{\alpha}\int_\mathbb{X}cq(x)dx}\\ 
           &=\frac{\frac{1}{\alpha}\int_{A}f(x)dx}{\frac{1}{\alpha}\int_\mathbb{X}f(x)dx}\\ 
           &=\int_A f(x)dx \quad (\mbox{since } \int_\mathbb{X}f(x)dx = 1)
\end{align*}
\end{itemize}
\section*{2.}
\begin{itemize}
\item[(a)] Let $g(x) = c(2x^{\theta -1} e^{-x} + x^{\theta - 1/2} e^{-x})$. Then 
\begin{align}
g(x) & = c\left(2\Gamma(\theta) frac{1}{\Gamma(\theta)}e^{\theta - 1} e^{-x} + \Gamma(\theta + 1/2) \frac{1}{\Gamma(\theta + 1/2)} e^{\theta -1/2}e^{-x}  \right)\notag\\
     & = c(2\Gamma(\theta) \mbox{Gamma}(\theta, 1) + \Gamma(\theta +1/2) \mbox{Gamma}(\theta +1/2, 1)). \label{eq1}
\end{align} 
Since $\int  g dx = 1$, hence 
\begin{equation}\label{eq2}
c = \frac{1}{2\Gamma(\theta) + \Gamma(\theta + 1/2)}.
\end{equation}
\item[(b)] and from \eqref{eq1}, $g$ is the mixture of 2 Gamma distribution  $\mbox{Gamma}(\theta, 1)$ and $\mbox{Gamma}(\theta +1/2, 1)$ with the corresponding weights 

\[
\frac{2\Gamma(\theta)}{2\Gamma(\theta) + \Gamma(\theta + 1/2)}, \quad \frac{\Gamma(\theta+1/2)}{2\Gamma(\theta) + \Gamma(\theta + 1/2)}
\]

\item[(c)] A procedure to sample from $g(x)$: 
\begin{itemize}
\item[1.] Generate $U \sim \mbox{Unif}(0,1), X_1 \sim \mbox{Gamma}(\theta, 1), X_2 \sim \mbox{Gamma}(\theta+1/2, 1)$ independently, where a procedure to sample from any $\mbox{Gamma}(., 1)$ distribution can be conducted using a rejection sampling procedure following from Theorem 3.3 (page 29 in the link and page 406 in the  book) \href{http://www.nrbook.com/devroye/Devroye_files/chapter_nine.pdf}{http://www.nrbook.com/devroye/Devroye\_files/chapter\_nine.pdf}    
\item[2.] If $U \leq \frac{2\Gamma(\theta)}{2\Gamma(\theta) + \Gamma(\theta + 1/2)}$ then $X = X_1$, otherwise, $X = X_2$
\item[3.]  The random variable $X$ has distribution $g(x)$.
\end{itemize}

\item [(d)] Sampling a random variable having distribution $f$ using $g$ as proposed distribution. 
It is easy to see 
\[
\sqrt{4+ x} \leq 2 + x^{1/2},
\]
hence
\begin{equation}\label{eq3}
q(x) \equiv \sqrt{4+x} x^{\theta -1} e^{-x} \leq (2+ x^{1/2}) x^{\theta-1} e^{-x }
\leq \frac{1}{c} g(x),
\end{equation}
where $c$ is defined as in \eqref{eq2}.
\begin{itemize}
\item[1.] Generate $X \sim g(x)$ as in part $(c)$, and generate $U \sim \mbox{Unif}(0,1)$ independently. Set $r(x) = \frac{c q(x)}{g(x)}$, where $q$ is defined as in \eqref{eq3}. 
\item[2.] If $U \leq r(X)$, then accept $X$, otherwise, repeat step 1.
\end{itemize} 
\end{itemize}
\end{document}
